 
\def\filename{leaflet-manual.tex}
\def\fileversion{v1.0d}   % change this when leaflet-manual changed, too.
\def\filedate{2012/06/04}
\def\docdate {2012/06/04} % change this when leaflet-manual changed, too.
\listfiles
\errorcontextlines=99
\documentclass[
notumble,
nofoldmark,
%%dvipdfm,
%%portrait,
%%titlepage,
%%nocombine,
%%a3paper,
%%debug,
%%nospecialtricks,
%%draft,
]{leaflet}
\setmargins{15mm}{8mm}{12mm}{12mm}
% \usepackage[widht=311mm,height=214mm,center]{crop}
\usepackage{tabularx}
\usepackage{lipsum}
\usepackage{eurosym}
\renewcommand*\foldmarkrule{.3mm}
\renewcommand*\foldmarklength{5mm}

\usepackage[T1]{fontenc}
% \usepackage[T3]{inputenc}
\usepackage{textcomp}
\usepackage{mathptmx}
\usepackage{libertine}
% \usepackage[scaled=0.9]{helvet}
\makeatletter
\def\ptmTeX{T\kern-.1667em\lower.5ex\hbox{E}\kern-.075emX\@}
\DeclareRobustCommand{\ptmLaTeX}{L\kern-.3em
        {\setbox0\hbox{T}%
         %\vb@xt@ % :-)
         \vbox to\ht0{\hbox{%
                            \csname S@\f@size\endcsname
                            \fontsize\sf@size\z@
                            \math@fontsfalse\selectfont
                            A}%
                      \vss}%
        }%
        \kern-.12em
        \ptmTeX}
\makeatother
\let\TeX=\ptmTeX
\let\LaTeX=\ptmLaTeX
\usepackage{shortvrb}
\MakeShortVerb{\|}
\usepackage{url}
\usepackage{graphicx}
\usepackage[dvipsnames,usenames]{color}
\usepackage[table]{xcolor}


% \usepackage{langscicolors}
%%%%\renewcommand{\descfont}{\normalfont}
\newcommand\Lpack[1]{\textsf{#1}}
\newcommand\Lclass[1]{\textsf{#1}}
\newcommand\Lopt[1]{\texttt{#1}}
\newcommand\Lprog[1]{\textit{#1}}

\newcommand*\defaultmarker{\textsuperscript\textasteriskcentered}


\title{Linked Data in Linguistics}
\author{%
  OWLG}
\date{Last updated~\docdate\\printed \today}

% \CutLine*{1}% Dotted line without scissors
% \CutLine*{6}%   

 
% \AddToBackground{3}{
%     \put(-2,10){
%         \includegraphics[width=.3\paperwidth]{somegraphic.png}%
%     } 
% }


\begin{document}

% position: | | |x|
%
% \maketitle
\thispagestyle{empty}
 
%%\tableofcontents
 

\vspace*{12cm}

\hspace*{-6mm}
\parbox{\textwidth}{
\sffamily
\parbox{\textwidth}{\Huge Linked Data \\ in Linguistics}\\ 
}
 

\newpage % position: |x| | |
\includegraphics[width=2\textwidth]{llod-cloud.png}
\section{Publications}
\begin{itemize}
\item ...
\end{itemize} 

\newpage % position: | |x| |
\vspace*{12.35cm} 
 \section{} %empty
 \parbox{\textwidth}{
...
 }

\newpage  % position: | | |x|

\section{About}

The Open Linguistics Working Group (OWLG) is an initiative of experts from different fields concerned with linguistic data. 
The aim of the OWLG is to provide and create the Linguistic Linked Open Data cloud (the LLOD-cloud) which enables an interoperable reuse of linguistic resources on the Web of Data.
This includes linguistic data from from various fields ranging from academic linguistics (e.g. typology, corpus linguistics) and applied linguistics (e.g. computational linguistics, lexicography, language documentation) to data from the Natural Language Processing and Semantic Web communities. 

The central goals of the OWLG are to:
\begin{enumerate}

\item Promote the idea of open data in linguistics and in relation to language data.
\item Develop the means for the representation of linguistic linked data.
\item Act as a central point of reference and support for those interested in open linguistic data.
\item Facilitate communication between researchers from different communities that use, distribute, or maintain open linguistic data.
\item Serve as a mediator between providers and users of technical infrastructure.
\item Build and maintain an index of open linguistic data sources and tools that link existing resources.
\item Assemble best-practice guidelines and use cases concerning creating, using and distributing data.
\item Provide guidance on legal issues surrounding linguistic data to the community.

\end{enumerate}
 
\newpage  % position: |x| | |
\section{Members}

\begin{itemize}
 \item ...
\end{itemize}

     
\newpage   % position: | |x| |
\section{Support us}
 
 \begin{itemize}
 \item Visit the homepage: http://linguistics.okfn.org/
 \item Subscribe to the mailing list: https://lists.okfn.org/mailman/listinfo/open-linguistics
 \item Have a look at the Wiki: http://wiki.okfn.org/Working_Groups/Linguistics
 \item Follow the blog: http://linguistics.okfn.org/blog/
\end{itemize}

\loggingall
\end{document}
\endinput
%%
%% End of file `leaflet-manual.tex'.
